\documentclass[12pt,ngerman,titlepage,a4paper, onesided]{article}

\usepackage{babel}

\begin{document}

%===========================================================
\begin{titlepage}
\begin{center}

\textbf{\LARGE Titel der Bachelorarbeit}
%bei langen Titeln, die mehrere Zeilen benoetigen:
%\textbf{\LARGE Langer Titel der  ueber \medskip mehrere Zeilen laeuft}

\bigskip\bigskip
\textbf{Bachelorarbeit}
%bei Masterarbeit:
%\textbf{Masterarbeit}

\bigskip\bigskip\bigskip
Vorgelegt von

\bigskip
\textbf{Vorname Nachname}

\bigskip
aus Geburtsort


\vfill
Angefertigt am\\
Mathematischen Institut\\ 
der Mathematisch-Naturwissenschaftlichen Fakult\"at\\ 
der Heinrich-Heine-Universit\"at D\"usseldorf

\bigskip
%Abgabedatum:
tt.\ Monat jjjj

\bigskip
Betreuer: Prof.\ Dr.\ Vorname Nachname
%Wird der Schwerpunkt der Abschlussarbeit im Anwendungsfach gewaehlt:
%Betreuer: Prof.\ Dr.\ Vorname Nachname\\
%Zweitbetreuer: Prof.\ Dr.\ Vorname Nachname

\end{center}
\end{titlepage}

\thispagestyle{empty}\mbox{}\pagebreak
\setcounter{page}{0}

%===========================================================
\tableofcontents
\pagebreak


%===========================================================
\section*{Einleitung}
\addcontentsline{toc}{section}{Einleitung}

In der vorgelegten Abschlussarbeit untersuchen wir ...


\pagebreak


%===========================================================
\section{Erster Abschnitt}

%===========================================================
\subsection{Ein Unterabschnitt}


%=========================================================== 
\section{Zweiter Abschnitt}


%===========================================================
\pagebreak
\begin{thebibliography}{ccccc}
\addcontentsline{toc}{section}{Literatur}

%So kann eine eine Monographie zitieren:
\bibitem{Arbarello et al 1985}
E.\ Arbarello, M.\ Cornalba, P.\ Griffiths, J.\ Harris:
Geometry of algebraic curves. I. 
Springer, New York, 1985.

%So kann man eine Originalarbeit zitieren:
\bibitem{Shepherd-Barron 1997}
N.\ Shepherd-Barron:
Fano threefolds in positive characteristic.
Compositio Math.\  105  (1997),  237--265.

\end{thebibliography}



%===========================================================
\pagebreak\noindent
\textbf{\LARGE Erkl\"arung}
\addcontentsline{toc}{section}{Erkl\"arung}

\bigskip\bigskip
\noindent 
Hiermit versichere ich, dass ich die   Bachelorarbeit selbst\"andig verfasst und keine
anderen als die angegebenen Quellen und Hilfsmittel benutzt habe.

\bigskip
\noindent
D\"usseldorf, den tt.\ Monat jjjj

\bigskip\bigskip\bigskip
\noindent
(Vorname Nachname)

\end{document}